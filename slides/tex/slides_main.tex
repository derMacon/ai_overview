\documentclass[10pt]{beamer}

% template preamble

\usetheme[subsectionpage=progressbar, progressbar=frametitle, sectionpage=none]{metropolis}

\usepackage{appendixnumberbeamer}

\usepackage{booktabs}
\usepackage[scale=2]{ccicons}

\usepackage{pgfplots}
\usepgfplotslibrary{dateplot}

\usepackage{xspace}
\newcommand{\themename}{\textbf{\textsc{metropolis}}\xspace}

% eigene packages 
\usepackage[english, ngerman]{babel}
\usepackage{blindtext}


% header
\title{Deeplearning}
\subtitle{Einführung - Thema 2}
\date{\today}
\author{Silas Hoffmann}
\institute{Fachhochschule Wedel}
\titlegraphic{\hfill\includegraphics[height=1.5cm]{_img/logo_alpha}}

\begin{document}

\maketitle

\begin{frame}
\frametitle{Inhalt}
\tableofcontents
\end{frame}

\part{Geschichtliche Entwicklung}

Die Entwicklung künstlicher Intelligenz ist momentan eines der wohl interessantesten Themen der heutigen technologischen Ära. Durch \emph{Machine Learning} ist es möglich viele Bereiche zu automatisieren die man dafür vorher nie in Betracht gezogen hatte. Dennoch hat es mich überrascht wie \glqq alt \grqq die Grundzügen dieser Technologie bereits sind. Die Anfänge finden sich schon Mitte der 50er Jahre als zum Beispiel diverse Experten auf diesem Gebiet das erste Mal zusammentrafen um im Zuge der \emph{Dartmouth Konferenz} über das Thema zu diskutieren. Aber auch bekannte Persönlichkeiten des Feldes, wie zum Beispiel \emph{Von Neumann} \footnote{Von Neumanns letztes Werk: zweiteiliges Manuskript über die damaligen Rechner und ihre Möglichkeiten das neuronale System zu emulieren.} erkannten früh die möglichen Zusammenhänge zwischen Mensch und Maschine.

Im folgenden Abschnitt werde ich etwas auf die geschichtlichen Aspekte von neuronalen Netzen eingehen. Hierbei werden insbesondere die generellen Aspekte der Funktionsweise von älteren Modellen bis hin zur aktuellen Entwicklung verfolgt. Ich werde versuchen die folgenden Leitfragen in diesem Abschnitt zu beantworten: 

\begin{itemize}
\item Woher kommt Deep Learning und wie ist dieser Begriff im Kontext zur künstlichen Intelligenz einzuordnen?
\item Welche Entwicklungen hat das Neuronale Netz von damals zu heute durchgemacht?
\end{itemize}


\import{geschichtliches/mcCullochPittsNeuron/}{mpNeuron.tex}
\clearpage

\import{geschichtliches/perceptor/}{perceptor.tex}
\clearpage

\import{geschichtliches/adeline/}{adeline.tex}
\clearpage

\import{geschichtliches/convolutionalNN/}{convolutionalnn.tex}
\clearpage

\import{geschichtliches/fazit/}{fazit.tex}
\section{Aktuelle Entwicklung}

\section{Backpropagation}

\subsection{Notation}

Um in einem mehrschichtigen Netz effizient die Kostenfunktion zu minimieren wird in vielen Fällen der \emph{Backpropagation}-Algorithmus verwendet. Dieser wurde bereits in den Siebzigerjahren definiert, erlangt jedoch erst im Jahr 1986 mit dem Paper \emph{Learning representations by back-propagating errors} von Rumelhart, Hinton und Williams Bekanntheit. 

Im vorherigen Teil habe ich bereits beschrieben was man unter dem Gradientenabstieg versteht und wie dieser auch bei mehrdimensionalen Funktionen (wie zum Beispiel der hierbei betrachteten Kostenfunktion) verwendet werden kann. Es wurde jedoch noch nicht vorgestellt, wie man auf ein mehrschichtiges Netz bezogen, diesen Gradienten überhaupt berechnen kann. 

Gemeinhin wird die Notation wie sie in Abbilung \ref{fig:weight_not} zu sehen ist, für ein Gewicht verwendet. In Abbildung \ref{fig:biasAct_not} steht das $b^l_j$ für den Schwellwert (\emph{Bias}) am Neuron mit dem Index \emph{j} im Layer mit dem Index \emph{l}. Selbes gilt für die Aktivierung welche mit einem \emph{a} gekennzeichnet wird. Mit den gegebenen Notationen können wir folgende Gleichung für die Aktivierung eines Neurons aufstellen (siehe Gleichung \ref{eq:act}). 

\begin{figure}[!htb]
	\centering
	\includegraphics[width=.9\linewidth]{img/weight_notation}
	\mycaption{Notation}{dlnielsen}
	\label{fig:weight_not}
\end{figure}

\begin{figure}[!htb]
	\centering
	\includegraphics[width=.4\linewidth]{img/biasAct_notation}
	\mycaption{Notation}{dlnielsen}
	\label{fig:biasAct_not}
\end{figure}

\begin{equation} \label{eq:act}
a^{l}_j = \sigma\left( \sum_k w^{l}_{jk} a^{l-1}_k + b^l_j \right),
\end{equation}

Diese Formel sollte bereits aus den vorherigen Kapiteln bekannt sein. Es wird hierbei lediglich eine Vektormultiplikation der beiden eingehenden Gewichtsvektors und Addition mit den Aktivierungsvektoren durchgeführt. Die generierte Ausgabe wird mit dem Schwellwert verrechnet und in eine Aktivierungsfunktion (wie sum Beispiel der Sigmoid-Funktion) gesteckt.

Um später einfacher mit all diesen Werten zu rechnen wird versucht die gegebenen Werte in eine Matrix- beziehungsweise Vektor-Darstellungsform zu bringen. Da Neuron mehrere ausgehende \myquote{Pfade} besitzen wird hierbei eine Matrix geformt. Das bisher beschriebene Gewicht $w^l_{jk}$ befindet sich hierbei in der Zeile mit dem Index \emph{j} und Spaltenindex \emph{k}. Da sich sowohl der Schwellwert als auch die Aktivierung lediglich auf ein einzelnes Neuron beziehen muss hierfür lediglich ein eindimensionaler Vektor pro Layer \emph{l} generiert werden. Die einzelnen Komponenten werden hierbei über den Index \emph{j} angesteuert. Die Notation für den Aktivierungsvektor der Schicht \emph{l} sieht dann derartig aus: $a^l_j$. Ähnliches gilt fur die Schwellwerte: $b^l_j$. 

Um mit diesen Vektoren arbeiten zu können muss man die abschließende Aktivierungsfunktion $\sigma$ vektorisieren 
\footnote{Klarer Verweis auf Nielsen Buch, Kapitel über Backpropagation im Detail \cite{dlnielsen}}. 
Dabei muss man die Funktion lediglich derartig umschreiben, dass diese auf die einzelnen Komponenten angewendet wird und nicht auf einen einzelnen Wert. Beispielsweise würde die Funktion $f(x) = x^2$ vektorisiert folgendermaßen aussehen: 

\begin{equation}
  f\left(\left[ \begin{array}{c} 2 \\ 3 \end{array} \right] \right)
  = \left[ \begin{array}{c} f(2) \\ f(3) \end{array} \right]
  = \left[ \begin{array}{c} 4 \\ 9 \end{array} \right],
\end{equation}

Mit diesen Zusammenfassungen kann nun die Gleichung \ref{eq:act} folgendermaßen umgeschrieben werden: 

\begin{equation}
  a^{l} = \sigma(w^l a^{l-1}+b^l).
\end{equation}

Diese Umschreibung abstrahiert das Denken über die lokalen Neuronen auf ein höheres Level sodass es einfacher ist das Gesamtbild zu betrachten. Um noch mehr Klarheit zu schaffen wird oftmals die Aktivierung einer Schicht \emph{l} aus der Funktion herausgezogen und mit dem Buchstaben \emph{z} versehen. $z^l$ kann nun in die Aktivierungsfunktion $\sigma$ eingesetzt werden. Im folgenden habe ich noch einmal alle bisherigen Erkenntnisse zusammengefasst: 

\begin{mytheo}{Backpropagation - Notation}{theoexample}

Bisherige Schreibweise:
\begin{equation}
  a^{l}_j = \sigma\left( \sum_k w^{l}_{jk} a^{l-1}_k + b^l_j \right) \nonumber
\end{equation}

Zusammengefasste Form:
\begin{equation}
a^l = \sigma(z^l)
\end{equation}

Gewichtete Eingabe:
\begin{equation}
  z^l \equiv w^l a^{l-1}+b^l
\end{equation}

\end{mytheo}

\subsection{Fundamentale Gleichungen}

Ziel des Backpropagation Algorithmus ist es herauszubekommen welche Gewichte und Schwellwerte verändert werden müssen um die Kostenfunktion zu minimieren. Im folgenden werde ich nach und nach die vier wichtigsten Formeln des Verfahrens beschreiben. 

\begin{mytheo}{Backpropagation - Fundamentale Gleichungen}{theoexample} \label{theo:zus}

\begin{equation} \label{eq:error}
\delta^L = \nabla_a C \odot \sigma'(z^L).
\end{equation}

\begin{equation}
\delta^l = ((w^{l+1})^T \delta^{l+1}) \odot \sigma'(z^l),
\end{equation}

\begin{equation}
\frac{\partial C}{\partial b^l_j} = \delta^l_j.
\end{equation}

\begin{equation}
\frac{\partial C}{\partial w^l_{jk}} = a^{l-1}_k \delta^l_j.
\end{equation}

\end{mytheo}


\subsection{Fehler auf der Ausgabeschicht (Gleichung \ref{eq:error})}
Um zu verstehen was man unter einem Fehler genau versteht sei angenommen die Ausgabe eines Neurons \emph{j} im Layer \emph{l} wird um einen unbestimmten Wert verzerrt. Mathematisch ausgedrückt sieht dies dann folgendermaßen aus: 

\begin{equation}
\sigma(z^l_j+\Delta z^l_j)
\end{equation}

Statt der herkömmlichen Ausgabe $\sigma{z^l_j}$ wird ein \emph{Fehler} $\Delta z^l_j$ hinzugefügt. Allgemein wird der Fehler eines einzelnen Neurons dadurch folgendermaßen beschrieben: 

\begin{equation}
\delta^l_j \equiv \frac{\partial C}{\partial z^l_j}.
\end{equation}

Wie wir bereits beim Gradientenabstieg gesehen haben, beschreibt diese partielle Ableitung die \glqq Steigung \grqq der Kostenfunktion im Bezug auf den die Ausgabe $z^l_j$. Dieser Wert dient im späteren Verlauf allerdings lediglich als Zwischenschritt, da wir im Endeffekt an den Gradienten bezüglich der Gewichte sowie den Schwellwerten interessiert sind. Im Nachhinein wird daher versucht den Fehler ${\partial C / \partial z^l_j}$ auf die Gewichte ${\partial C / \partial w^l_j}$ und Bias ${\partial C / \partial b^l_j}$ zurückzuführen. 

Die komponentenweise Darstellung für die Berechnung des Fehlers auf der Ausgabeschicht $\delta^L$ sieht folgendermaßen aus:

\begin{equation}
\delta^L_j = \frac{\partial C}{\partial a^L_j} \sigma'(z^L_j)
\end{equation}

Der vordere Teil $\partial C / \partial a^L_j$ beschreibt wie sich die Kostenfunktion bezüglich eines Neurons mit dem Index \emph{j} auf der Ausgabeschicht verhält (Steigung der Kostenfunktion nach $a^L_j$). Wenn das Neuron nicht sehr viel \emph{Einfluss} auf die Kostenfunktion nimmt, bleibt dieser Faktor klein und das Ergebnisfehler fällt minimal aus. Der hintere Teil beschreibt die Ableitung der Aktivierungsfunktion und die Steigung an der gegebenen Stelle $z^L_j$. 
Da man hierbei auch wieder versuchen möchte die Formeln möglichst allgemein zu halten wird der Fehler auf wieder als Vektor der Ausgabeschicht definiert. Das sieht dann so aus: 

\begin{equation}
\delta^L = \nabla_a C \odot \sigma'(z^L)
\end{equation}

Dies ist auch die Darstellung wie sie Anfang des Abschnitts verwendet wurde (siehe \ref{theo:zus}). $\nabla_a C$ stellt dabei den Vektor dar dessen partielle Ableitung $\partial C / \partial a^L_j$ entsprechen. Es ist also lediglich eine andere Schreibweise für die partielle Ableitung bezogen auf die komplette Ausgabeschicht. Wenn wir für die Kostenfunktion wie schon in früheren Kapiteln die quadratische Kostenfunktion wählen entspricht \emph{C} gleich $C = \frac{1}{2} \sum_j (y_j-a^L_j)^2$. Nach $a^L_j$ abgeleitet entspricht dies dann wiederum $\partial C / \partial a^L_j = (a_j^L-y_j)$. Die Herleitung ist sehr ähnlich zu der des Gradienten (siehe \ref{deri:grad}), hier fehlt lediglich die Iteration über die kompletten Trainingsdatensätze. 

Der entstandene Vektor ($\nabla_a C$) wird dann komponentenweise mit der Ableitung der Aktivierungsfunktion multipliziert. Da diese vektorisiert wurde entspricht das Ergebnis von $\sigma'(z^L)$ ebenfalls wieder einem Vektor was diese Rechenoperation ermöglicht. Eine andere Schreibweise ist daher folgende Gleichung: 

\begin{equation}
\delta^L = (a^L-y) \odot \sigma'(z^L)
\end{equation}

Hierbei wurde lediglich die Ableitung eingesetzt. 
\subsection{Convolutional Neural Network}

\begin{frame}
\frametitle{Biologische Zellarten}

\begin{figure}
	\includegraphics[width=.7\linewidth]{./aktuelleEntwicklung/convolutionalNN/img/simpleVsComplex_alpha}
\end{figure}


\note[item]{1962: zwei Neurophysiologen Torsten Wiesel und David Hubel}

\note[item]{Konzept der simple und complex cells
\begin{itemize}
    \item nicht positionsbunden - spatial invariance, räumliche Invarianz
\end{itemize}}

\note[item]{Arten von Zellen zur Erkennung einfacher Kanten und Balken
\begin{itemize}
    \item \emph{simple cells}: ist Positionsgebunden
    \item \emph{complex cells}: Muster können an beliebigen Positionen auftauchen
\end{itemize}}

\note[item]{1962: Konzept wie im Bild}
\note[item]{1980er Dr. Kunihiko Fukushima: erstes Modell nach diesem Konzept}

\end{frame}


\begin{frame}
\frametitle{Anfänge}

\begin{itemize}
\item Yann LeCun: erstes Modell zum Erkennen von Handschrift
\item \emph{Verwendung von MNIST database of handwritten digits}
\begin{itemize}
	\item 60.000 Trainingsdatensätze
	\item 10.000 zum Berechnen des Fehlers
\end{itemize}
\end{itemize}

\begin{figure}
	\includegraphics[width=.9\linewidth]{./aktuelleEntwicklung/convolutionalNN/img/cnn_overview_alpha}
\end{figure}


\note[item]{Pioniere, fr. Informatiker Yann LeCun}
\note[item]{Bekannteste Ausarbeitung über CNN für Handschriften}
\note[item]{\emph{Verwendung von MNIST database of handwritten digits}
\begin{itemize}
	\item 60.000 Trainingsdatensätze
	\item 10.000 zum Berechnen des Fehlers
    \item unterschiedliche Personen für Trainings- und Evaluierungsdatensätze
\end{itemize}}

\note[item]{soll erkennen ob ein Bild zu einer (oder mehreren) bestimmten Klasse(n) gehört
\begin{itemize}
    \item von \emph{low-level} Eigenschaften auf komplexe Formen schließen
\end{itemize}}

\note[item]{Covolutional NN: zwei wesentliche Komponenten
\begin{itemize}
    \item \emph{Convolutional layer}: Filter
    \item \emph{Pooling Layer}: Aggregations-Schichten
    \item wiederholen sich abwechselnd
\end{itemize}}

\end{frame}


\begin{frame}
\frametitle{Convolutinal Layer - Filter}

\begin{itemize}
\item Mehrdimensionales Array mit Farbwerten zur Repräsentation im Rechner
\item Durch Filter auf bestimmte \emph{Low-Level} Eigenschaften schließen
\end{itemize}

\begin{figure}
	\includegraphics[width=.8\linewidth]{./aktuelleEntwicklung/convolutionalNN/img/cnn_convLayer_alpha}
\end{figure}


\note[item]{Array als Eingabe
\begin{itemize}
    \item Repräsentiert die Pixel im Bild
\end{itemize}}

\note[item]{Farbwertearray kann pro Pixel mehrere Werte enthalten
\begin{itemize}
    \item entsprechend eventuell auch mehrere Dimensionen im Array
\end{itemize}}

\note[item]{Fenster \emph{läuft} Eingabematrix ab
\begin{itemize}
    \item dadurch simple Formen erkennen
    \item Beispiel folgt
\end{itemize}}

\note[item]{Hidden Layer kann als Ansammlung von low-level Merkmalen verstanden werden}

\end{frame}


\begin{frame}
\frametitle{Filter}


\begin{itemize}
\item Generell

\begin{itemize}
	\item Besitzt feste Pixelgröße (\emph{Kernelsize}) \& Schrittweite
	\item Scannt Bild Zeilenweise
	\item \emph{Padding} legt Verfahren für Rand des Bildes fest
	\item Ausgabe wird \emph{activation} oder \emph{feature map} genannt
\end{itemize}

\item Praxis
\begin{itemize}
	%\item \emph{Convolutional Layer} mit 32 oder 16 Bit
	\item Jeder Filter generiert eigene Ausgabematrix
	\item Nächster Convolutional Layer verwendet Ausgabematrizen als Input
	\item Ausgabe wird in \emph{Pooling Layer} gesteckt
\end{itemize}

\end{itemize}


\note[item]{Bsp. Filter 2 x 2, Schrittweite: 2 - führt zu Halbierung der InputMatrix
\begin{itemize}
    \item Im Bsp. hängen immer 4 Pixel an einem Filter, die Eingabematrix wird gefaltet (convolute)
\end{itemize}}

\note[item]{Filter generieren eigene Ausgabematrix}
\note[item]{Filter können auch auf Filter folgen}
\note[item]{von Filtern generierte Ausgaben werden auch \emph{activation map} oder \emph{feature map} genannt}

\end{frame}


\begin{frame}
\frametitle{Filter - Funktionsweise}

\begin{figure}
	\includegraphics[width=\linewidth]{./aktuelleEntwicklung/convolutionalNN/img/filter}
\end{figure}

\note[item]{Bild erläutern
\begin{itemize}
    \item Beispiel: Ziffer 7
    \item Strich am oberen Rand
    \item Gewichtsmatrix hier getrennt aufgeführt
    \item rot: negative Werte
    \item grün: positive Werte
\end{itemize}}

\note[item]{Dieses Feature (oberer Strich) kann aber auch bei anderen Ziffern auftauchen
\begin{itemize}
    \item Bsp. schlecht geschriebene Ziffer Null
\end{itemize}}

\note[item]{Erkannte Merkmale können von weiteren Filtern genutzt werden
\begin{itemize}
    \item erinnert an ganz alte Prinzipien 
    \item wie schonn beim Adeline Modell, hier jedoch mit mehreren Schichten
\end{itemize}}

\end{frame}


\begin{frame}
\frametitle{Pooling Layer}

\begin{itemize}
\item Aggregiert die Ergebnisse von Convolutional Layern
\item Ziele
\begin{itemize}
	\item Nur die relevantesten Signale an nächste Schicht weitergeben
	\item Anzahl der Parameter im Netz reduzieren
\end{itemize}

\item \emph{MaxPooling Layer} am weitesten verbreitet
\end{itemize}



\note[item]{Pooling Layer
\begin{itemize}
    \item aggregiert Ergebnisse von Convolutional Layern
    \item Zweck: nur die relevantesten Signale an die nächste Schicht weitergeben 
\end{itemize}}

\note[item]{\emph{während die Größe des Inputs durch die Faltungen und das Pooling immer weiter reduziert wird, erhöht sich die Anzahl der Filter zur Erkennung von übergeordneten Signalen zunehmend}}

\note[item]{Verschiedene Pooling-Mechanismen:
\begin{itemize}
    \item MaxPooling: 
    \begin{itemize}
        \item am weitesten verbreitet
        \item maximale Eingabewert wird weitergegeben
    \end{itemize}
    \item fractional max pooling
    \item lp pooling
    \item mean pooling
    \item stochastic pooling 
    \item spatial pooling
    \item generalized pooling
\end{itemize}}

\end{frame}



\begin{frame}
\frametitle{Fully Connected Layer}

\begin{itemize}
\item Ausgagngspunkt: \emph{High-Level} Merkmale bereits durch frühere Schichten erkannt 
\item Alle Neuronen der Ausgabeschicht sowie dieser Merkmale alle direkt miteinander verbunden
\item Ausgabe sollte mit den richtigen Gewichten / Schwellwerten relativ eindeutige Ausgaben generieren
\end{itemize}

\begin{figure}
	\includegraphics[width=.9\linewidth]{./aktuelleEntwicklung/convolutionalNN/img/cnn_overview_alpha}
\end{figure}


\note[item]{auch \emph{dense Layer} genannt}
\note[item]{Ausgagngspunkt: \emph{High-Level} Merkmale bereits durch frühere Schichten erkannt
\begin{itemize}
    \item Neuronen halten diese Eigenschaften
\end{itemize}}

\note[item]{Ausgabeneuronen repräsentieren verschienden Klassen
\begin{itemize}
    \item siehe Klassifizierungsproblem
    \item Fully connected Layer: stellt verbindung zwischen letztem hidden Layer und Ausgabelayer bereit
\end{itemize}}

\note[item]{Beispiel: Schnörkel zu Ziffern interpretieren
\begin{itemize}
    \item 10 dimensionaler Ausgabevektor bei Ziffern
\end{itemize}}

\end{frame}






\begin{frame}[standout]
  Questions?
\end{frame}

\appendix

\begin{frame}[fragile]{Backup slides}
  Sometimes, it is useful to add slides at the end of your presentation to
  refer to during audience questions.

  The best way to do this is to include the \verb|appendixnumberbeamer|
  package in your preamble and call \verb|\appendix| before your backup slides.

  \themename will automatically turn off slide numbering and progress bars for
  slides in the appendix.
\end{frame}

\begin{frame}[allowframebreaks]{References}

\nocite{*} 

  \bibliography{verweise}
  \bibliographystyle{abbrv}

\end{frame}


\end{document}