\section{Multilayer Perceptron}

Eines der bekanntesten Architekturen der neuronalen Netze ist das sogenannte \emph{Multilayer Perceptron}. Diese sehr vielfältig eingesetzte Struktur verwendet den erläuterten Backpropagation-Lernalgorithmus und erzielt damit in der Regel sehr gute Ergebnisse. Beim MLP spielt die Aktivierungsfunktion durchaus auch eine tragende Rolle und ist nicht fest definiert (häufig wird jedoch die Sigmoid-Funktion verwendet). Wenn das Netzwerk sehr viele verborgene Zwischenschichten besitzt, wird es allgemein als \emph{tief} bezeichnet und es kann zu Problemen beim Training kommen. Hierbei gibt es spezielle Techniken die diese Probleme lösen können und generell unter dem Namen \emph{deep learning} zusammengefasst werden. Da diese Architektur als so vielfältig gilt, kann sie in vielen verschiedenen Anwendungsbereichen eingesetzt werden. Hier ein paar Beispiele:

\begin{multicols}{2}
\begin{itemize}
\item Mustererkennung
\item Funktionenapproximation
\item Klassifizierung
\item Prognose
\item Diagnose
\item Steuerung
\item Optimierung
\end{itemize}
\end{multicols}

\paragraph{Zusammenfassung - Aufbau}

Da diese Architektur als Grundlage für die Erklärung des Backpropagation-Algorithmus gewählt wurde, folgt nur eine kurze Zusammenfassung des Aufbaus. Für eine ausführliche Erklärung sämtlicher Notations-Regeln etc. siehe Abschnitt \ref{ss:notationBP}.

Diese Struktur gilt als mehrschichtige vorwärtsgekoppeltes Netzwerk. Teilweise wird sie auch als Mehrschichten-Perzeptron bezeichnet, weil sie einem Perzeptron-Netzwerk mit mehreren Schichten sehr ähnelt. Dieses Netzwerk besteht aus einer Eingabeschicht, einer oder mehrerer innerer Schichten\footnote{in englisch auch häufig \emph{hidden Layers} genannt} und einer Ausgabeschicht. 

\begin{figure}[!htb]
	\centering
	\includegraphics[width=.8\linewidth]{./img/mlp1}
	\mycaption{MLP - Aufbau}{ss}
	\label{fig:mlpAufbau}
\end{figure}

Das Netz kann \emph{n} Eingabewerte annehmen, welche jeweils mit den Neuronen $x_1$ bis $x_n$ entgegengenommen werden. Die einzelnen Neuronen werden mit einer Aktivierungsfunktion belegt und verwenden intern einen eigenen Schwellwert. Das durch-Iterieren diverser Eingabewerte wird auch als \emph{Feedforward} bezeichnet, wobei am Ende \emph{m} Ausgabewerte entstehen, die über die Ausgabeneutronen der letzten Schicht weitergeleitet werden (siehe $z_1$ bis $z_n$).

\FloatBarrier

\paragraph{Aktivierungsfunktionen}

\subparagraph{Sigmoid}
Um beim Backpropagation-Algorithmus die ersten Ableitungen der Aktivierungsfunktion berechnen zu können, müssen diese stetig differenzierbar sein. Eine häufig verwendete Funktion ist die \emph{Sigmoid}-Funktion. Diese logistische Funktion zeichnet sich dadurch aus, dass alle möglichen Eingabewerte einen Ausgabewert zwischen 0 und 1 generieren. Sehr kleine Werte streben dabei gegen 0, während sehr große Werte gegen 1 streben. Die Formel sieht folgendermaßen aus:

\begin{equation}
f(x) = \frac{1}{1 + exp(-b * x)}
\end{equation}

Die Konstante \emph{b} beschreibt dabei die Steilheit der Kurve. Die Ableitung der Funktion besitzt die Form $f'(x) = b * f(x)(1-f(x))$.

\begin{figure}[!htb]
	\centering
	\includegraphics[width=.8\linewidth]{./img/plotSigmoid}
	\mycaption{Sigmoid - Plot}{ss}
	\label{fig:mlpSigmoid}
\end{figure}



% \subparagraph{Tangens Hyperbolicus}

% Die Funktion wird auch häufig als Aktivierungsfunktion verwendet und besitzt die Gleichung sowie die Ableitungen:

% \begin{equation}
% f(x) = tanh(x) = \frac{2}{1+exp(-2x}-1
% \end{equation}

% \begin{equation}
% f'(x) = 1 - tanh^2(x)
% \end{equation}
