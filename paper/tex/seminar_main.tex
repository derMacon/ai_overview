\documentclass[a4paper, 11pt]{scrartcl}

%Packages fuer die Darstellung von Quellcode
\usepackage{beramono} %schoene Schriftart
\usepackage{color}
\usepackage[dvipsnames]{xcolor}
\usepackage{listings}

\usepackage[utf8]{inputenc}
\usepackage[english, ngerman]{babel}
\usepackage[T1]{fontenc}
\usepackage{lmodern}

\usepackage{enumitem}
%\usepackage[a4paper, total={6in, 8.5in}]{geometry}
\usepackage[a4paper, bottom=2cm,includeheadfoot]{geometry}
\usepackage{afterpage} %Um Titelseite mit gesonderter Formatierung zu belegen
\usepackage{blindtext}
\usepackage{todonotes}

% Listen mit mehreren Spalten
\usepackage{multicol}

% Math
\usepackage{amsmath}
\usepackage{eqnarray}

% Schoene Tabellen 
\usepackage{booktabs}
\usepackage{longtable}

% Um auf Ueberschriften und dergleichen namentlich verweisen zu koennen
\usepackage{nameref}

% Um Dateien einzubinden
\usepackage{embedfile}

% Extra Platz nach eigenem Ermessen via \xspace
% \usepackage{xspace}


% Noetig fuer verschachtelte Imports 
\usepackage{import}
% \usepackage{subcaption} % Fehler in Kombi mit Subfig 
% Noetig fuer 2 x 2 Bilder in einer Figureumgebung
\usepackage{mwe}

% Bilder
\usepackage{graphicx}
% Standardbreite um Text und Bilder gem. darzustellen
\setkeys{Gin}{width=0.8\linewidth}
% mehrere Bilder nebeneinander
% \usepackage{minipage}
\usepackage{subfig}

% Floatbarrier
\usepackage{placeins}

% Einfacheres Navigieren im PDF/Web
\usepackage{url}
% Verweise jetzt nicht nur auf Ueberschrift sondern auf label
%\usepackage[all]{hypcap} 
\usepackage[hidelinks]{hyperref}
\hypersetup{
pdfstartview={FitH}, %Doc fuellt automatisch die horizontale Achse aus
pdftitle={Einführung DeepLearning, Thema 2},
pdfauthor={Silas Hoffmann, inf103088}
}



% Header Informationen festgelegt
\usepackage{fancyhdr}
\pagestyle{fancy}
\fancyhf{}
\rhead{\rightmark}
\chead{\thepart}
\lhead{\nouppercase{\leftmark}}
\cfoot{\thepage}


%Quellcodestyle Spezifikationen
\definecolor{DarkPurple}{rgb}{0.4,0.1,0.4}
\definecolor{DarkCyan}{rgb}{0.0,0.5,0.4}
\definecolor{LightLime}{rgb}{0.3,0.5,0.4}
\definecolor{Blue}{rgb}{0.0,0.0,1.0}




% ---------- Macros -----------

\newcommand{\mycaption}[2] {
	\caption[#1]{#1 \cite{#2}}
}

% Macro um nicht jedes Mal die listing und namen-Referenz angeben zu muessen
\newcommand{\lstref}[1]{(siehe Listing \ref{#1}, S. \pageref{#1}, \nameref{#1})}

% -----------------------------

% Syntaxhighlighting festgelegt
\lstdefinestyle{CodeHighlighting} 
{
language=Python, %mit mehreren Sprachen moeglich, ermoeglicht Syntaxhighlighting
columns=flexible,
numbers=left,
frame=single,
frameround=tttt,
showstringspaces=false,
basicstyle=\footnotesize\ttfamily,
keywordstyle=\bfseries\color{DarkPurple},
commentstyle=\itshape\color{LightLime},
stringstyle=\color{Blue}
}


% Titelseite
\title{\includegraphics[width=\linewidth]{img/dl_logo}}
% \title{Einführung in DeepLearning}
\author{Silas Hoffmann, inf103088
\normalsize
\\5. Fachsemester
\\6. Verwaltungssemester}
\date{\today}

\newcommand{\prefWidth}{width=0.8\linewidth}

\begin{document}
\newgeometry{a4paper, bottom=6cm}
\maketitle

\vfill
\begin{center}
	Thema 2
	
	\vspace{0.5cm}
	\Large
	\textbf{Seminar}
	\normalsize
	
	\vspace{0.3cm}
	im Sommersemester 2020
	
	\vspace{0.5cm}
	Dozent: Prof. Dr. Dennis Säring
	
	\vfill
	\small
	Fachbereich Informatik
	
	\vspace{0.3cm}
	Fachhochschule Wedel
\end{center}
\thispagestyle{empty}
\restoregeometry

% Inhaltsverzeichnis
\newpage
\pagestyle{plain}
\tableofcontents
\newpage
% \bibliography{verweise}
\listoffigures
%\pagestyle{main}
\newpage

\pagestyle{fancy}



\part{Einführung}

\blindtext



\clearpage
\part{Geschichtliche Entwicklung}

Die Entwicklung künstlicher Intelligenz ist momentan eines der wohl interessantesten Themen der heutigen technologischen Ära. Durch \emph{Machine Learning} ist es möglich viele Bereiche zu automatisieren die man dafür vorher nie in Betracht gezogen hatte. Dennoch hat es mich überrascht wie \glqq alt \grqq die Grundzügen dieser Technologie bereits sind. Die Anfänge finden sich schon Mitte der 50er Jahre als zum Beispiel diverse Experten auf diesem Gebiet das erste Mal zusammentrafen um im Zuge der \emph{Dartmouth Konferenz} über das Thema zu diskutieren. Aber auch bekannte Persönlichkeiten des Feldes, wie zum Beispiel \emph{Von Neumann} \footnote{Von Neumanns letztes Werk: zweiteiliges Manuskript über die damaligen Rechner und ihre Möglichkeiten das neuronale System zu emulieren.} erkannten früh die möglichen Zusammenhänge zwischen Mensch und Maschine.

Im folgenden Abschnitt werde ich etwas auf die geschichtlichen Aspekte von neuronalen Netzen eingehen. Hierbei werden insbesondere die generellen Aspekte der Funktionsweise von älteren Modellen bis hin zur aktuellen Entwicklung verfolgt. Ich werde versuchen die folgenden Leitfragen in diesem Abschnitt zu beantworten: 

\begin{itemize}
\item Woher kommt Deep Learning und wie ist dieser Begriff im Kontext zur künstlichen Intelligenz einzuordnen?
\item Welche Entwicklungen hat das Neuronale Netz von damals zu heute durchgemacht?
\end{itemize}


\import{geschichtliches/mcCullochPittsNeuron/}{mpNeuron.tex}
\clearpage

\import{geschichtliches/perceptor/}{perceptor.tex}
\clearpage

\import{geschichtliches/adeline/}{adeline.tex}
\clearpage

\import{geschichtliches/convolutionalNN/}{convolutionalnn.tex}
\clearpage

\import{geschichtliches/fazit/}{fazit.tex}

\clearpage
\part{Aktuelle Entwicklung}

Im folgenden Abschnitt werde ich auf die aktuelle Entwicklung der neuronalen Netze eingehen. Insbesondere wird dabei auch noch einmal der Lernalgorithmus (Backpropagation) dieser Netze beschrieben und wofür welche genauen Arten von Architekturen verwendet werden. Die Mathematik hierbei ist durchaus relativ fortgeschritten, sodass sich ein Blick in das frei (kostenlos) verfügbare E-Book \emph{Neural Networks and Deep Learning} von Michael Nielsen \cite{dlnielsen} sehr lohnt. Daraus resultieren folgende Leitfragen: 

\begin{itemize}
\item Wo können diese Verfahren eingesetzt werden, bzw. wo werden diese bereits eingesetzt?
\item Welche Ansätze und Architekturen sind zur Zeit state-of-the-art?
\end{itemize}

\import{aktuelleEntwicklung/backpropagation/}{backpropagation.tex}
\clearpage

\import{aktuelleEntwicklung/multilayerPerceptron/}{multilayerPerceptron.tex}
\clearpage

\import{aktuelleEntwicklung/recurrentNN/}{recurrentNN.tex}
\clearpage

\import{aktuelleEntwicklung/fazit/}{fazit.tex}
\clearpage



 
\clearpage
\appendix
\pagenumbering{roman}
\thispagestyle{plain}

% alle bibtex Referenzen aufzählen, auch wenn nicht verwendet
\nocite{*} 

\bibliographystyle{plain}
\bibliography{verweise}

\clearpage
\section{Anhang}
\import{geschichtliches/mcCullochPittsNeuron/}{mpNeuron_anhang.tex}
\FloatBarrier




\end{document}
